% Datei: Documents/latex/vimlatex/main.tex
% Begonnen: 20.07.2014 
% Änderungen siehe git log 
%
% Anleitung für Latex-Suite

\documentclass[a4paper,parskip=half,draft=true,DIV=15]{scrartcl}

\usepackage[utf8]{inputenc}
\usepackage[T1]{fontenc}
\usepackage[ngerman]{babel}
\usepackage{listings}

% Die Option dvipdfmx und das Paket bmpsize verhindern die Fehlermeldung von
% latex: "cannot determine size of graphic"
\usepackage[dvipdfmx]{graphicx}
\usepackage{bmpsize}

\usepackage{apacite}

\usepackage{lmodern}	% Schrift "Latin Modern"
\usepackage{hyperref}
\hypersetup{colorlinks=true, linkcolor=blue, urlcolor=blue}
%\pagestyle{empty}	% Auskommentieren, wenn Seitenzahlen gewünscht

\newcommand{\LS}{Latex-Suite}
\newcommand{\LSR}{\LS{} Reference}
\newcommand{\vimhNP}[1]{\texttt{h: #1}}
\newcommand{\vimh}[1]{(\vimhNP{#1})}

% Angabe für den Titel
\title{\LS{} -- die ultimative Anleitung}
\author{Marco Strehler}
\date{04.11.2014}

\begin{document} 

\maketitle

\section{Einführung}

In der Referenz zur \LS{} ist zwar alles vorhanden, aber vieles wird
nur sehr knapp erwähnt und mit verschiedenen Problemen wird der User alleine
gelassen.

Diese Anleitung soll es dem Einsteiger in die \LS{} einfach machen und
ist aus den Notizen entstanden, die ich mir -- zur Ergänzung zur Reference --
selbst gemacht habe.

Konventionen in diesem Dokument:

\begin{itemize}
  \item vim-Hilfe: immer wenn auf das Hilfe-System von vim hingewiesen wird
    (aufgerufen aus dem Normal-Modus mit dem Befehl \vimhNP{help}, wobei
    \texttt{help} hier das gesuchte Hilfethema ist). Wird genau diese Notation
    gewählt: also z.B. \vimhNP{latex-suite} oder \vimhNP{tutorial}.
  \item Die Einstellungen werden jeweils als kleines Listing dargestellt.
\end{itemize}

\section{Konfigurations-Dateien}

Wie es sich für eine richtige Terminal-Applikation gehört, müssen alle Einstellungen in Text-Files vermerkt werden.
Um die Sache zudem nicht allzu einfach zu machen, muss man seine Konfigurationen in verschiedenen Dateien vornehmen.
Was ebenfalls nett zu wissen ist, z.T. existieren die Dateien noch nicht und müssen von Hand selbst angelegt werden.
Das ist zwar nicht schwierig, aber wenn man sich darauf verlässt, dass allenfalls schon etwas da ist, dann verzweifelt man.
Die Dateien und die Einstellungen im Überblick:

\begin{description}
  \item[vimrc] Konfigurationsfile für vim \vimh{vimrc}.
  \item[tex.vim] Konfigurationsfile für globale Variablen der \LS.
    Pfad: \texttt{\textasciitilde/.vim/ftplugin/tex.vim}. \textbf{Datei existiert nicht, muss von Hand erzeugt werden.}
  \item[tex\_macros.vim] Definitionen für Macros (IMAP).
    Pfad: \texttt{\textasciitilde/.vim/after/ftplugin/tex\_macros.vim}. \textbf{Datei existiert nicht, muss von Hand erzeugt werden.}
	\item[bib.vim] Konfiguraitonsfile für bib-Files.
	  Pfad: \texttt{\textasciitilde/.vim/ftplugin/bib.vim}.
\end{description}


\section{Templates}

\subsection{Templates einfügen}

Siehe \vimhNP{latex-suite-templates}.

Ein neues LaTeX-Dokument beginnt man am besten mit einem Dokument-Gerüst
(Template). Das Vorgehen ist dabei folgendermassen:

\begin{enumerate}
  
  \item In vim ein neues Dokument erstellen. Wichtig ist dabei die Endung
    \texttt{.tex}, damit die \LS{} aktiviert wird.

  \item mit \texttt{:TTemplate<CR>} eine Liste mit den vorhandenen
    Vorlagen im oben erwähnten Template-Ordner anzeigen lassen. Mit der
    Eingabe der entsprechenden Nummer und <CR> wird das Template an der
    Cursorstelle eingefügt. Alternativ kann der Template-Namen gleich angegeben
    werden. Also z.B. \texttt{:TTemplate meinTemplate}. Gut zu wissen: die
    Auto-Vervollständigung mit der Tabulator-Taste funktioniert hier ebenfalls.
    Mit \texttt{:TTe<Tab>} sowie \texttt{m<Tab>} bekommt man mit wenigen Tastenschritten
    \texttt{:TTemplate meinTemplate}.

\end{enumerate}

Bei mir funktioniert das Template-Menü in MacVim nicht korrekt. Genauer: die Nummern der Templates
wird angezeigt, allerdings nicht der Template-Name, was das Menü praktisch unbrauchbar macht.
Da ich aber sowieso der Meinung bin, dass vim möglichst über die Tastatur bedient werden soll,
ist das ein verschmerzbarer Bug.

\subsection{Templates anpassen, eigene Templates einfügen}

Die LS-Referenz gibt an, dass die Templates im Verzeichnis \texttt{\$VIM/ftplugin/latex-suite/templates} zu finden sind.
Das ist aber ein Verzeichnis, dass es so nicht gibt.
Dabei ist \texttt{\$VIM} das Kürzel für das Benutzerverzeichnis, das auf unterschiedlichen Betriebssystemen unterschiedliche Pfade aufweist.

Auf meinem Mac gelange ich aus dem Terminal mit dem Befehl

\texttt{cd \textasciitilde/.vim/ftplugin/latex-suite/templates/}

dorthin. Wobei das Zeichen \textasciitilde{} (Tilde), wiederum eine Abkürzung für das Userverzeichnis ist.
Erzeugt wird es mit cmd-N.

Die Referenz erwähnt weiter, dass Templates \emph{dynamic elements} enthalten können in der Form von Vim-Ausdrücken.
Leider schweigt sich die Referenz darüber aus, wie das zu bewerkstelligen ist.
Die vorhandenen Beispiele scheinen Befehle zwischen zwei \texttt{!comp!}-Ausdrücken zu platzieren.
Siehe Listening \ref{report}.

Weitere Informationen zu vim-Expressionen holt man sich mit der internen Hilfe mit den Befehlen
\texttt{:h expression} oder
\texttt{:h usr\_41.txt}.

Das bereits vorhandene Beispiel im Ordner \texttt{templates}:

\lstset{%
	showtabs=false,
      	numbers=left,
	numberstyle=\tiny,
	breaklines=true,
      	breakautoindent=true,
      	basicstyle=\small,
	language=[LaTeX]TeX,
      	keywordstyle=\bfseries,
      	stringstyle=\ttfamily}

\lstinputlisting[caption={Template report.tex},label=report]{/Users/marcostrehler/.vim/ftplugin/latex-suite/templates/report.tex}

Wenn beim Templete Zeileneinzüge verwendet werden, dann kommt es zu einem hässlchen Treppeneffekt.

\subsection{Hinweise zu eigenen Templates}

Beim Erstellen von eigenen Templates stellen sich m.E. zwei grundsätzliche Fragen:
\begin{itemize}
  \item Wie viel soll ich in ein Template hineinpacken? Das meint, wie viele Strukturen sollen schon fertig für den
    Gebrauch in so einer Vorlage vorhanden sein (z.B. schon einige Frame-Umgebungen bei der Beamer-Klasse).
  \item Soll ich für eine Dokumentklasse verschiedene Templates mit verschiedenen Optionen als separate Dateien
    abspeichern oder soll ein Template mit verschiedenen Optionen (zum Beispiel mit der Möglichkeit des Auskommentieren)
    zur Verfügung gestellt werden?
\end{itemize}

Ich empfehle \emph{wenig} Strukturen (also kurze Dateien). Insbesondere weil für mich repetivie Codes in ein Makro gehören.

Aus ähnlichen Gründen versuche ich möglichst, Optionen als Auskommentierte Alternativen zu gestalten.
Das reduziert die vorhandenen, und zu pflegenden Templates.

Also möglichst \emph{kurze} und \emph{wenig} Templates erzeugen.

Am besten speichert man sich ein persönlicher Template-Header als Makro ab (Listing \ref{lst:makro}) und kann ihn so in schon vorhandene Templates bequem einfügen.
In das Templates können dann noch die von der \LS{} unterstützten Platzhalter \texttt{<+Platzhalter+>} eingebaut werden.
Mit Ctrl-J kann dann bequem von Platzhalter zu Platzhalter gesprungen und die wichtigsten Daten eingegeben werden.

\begin{lstlisting}[float,caption={Template-Header, wie er als Makro verwendet werden kann.},label=lst:makro]
<+	+>	!comp!	!exe!
% Datei: !comp!expand("%")!comp!
% Begonnen: !comp!strftime("%d.%m.%Y ")!comp!
% Letzte Aenderung: !comp!strftime("%d.%m.%Y ")!comp!
\end{lstlisting}

In vim können alle Templates zur Bearbeitung mit dem Befehl

\texttt{:arga ~/.vim/ftplugin/latex-suite/templates/*.tex}

geöffnet und in die Argumenten-Liste geladen werden.
So können alle (schon vorhandenenen Templates) an die \LS{} angepasst werden.
Dabei hilft ein Template-Header, der in ein Makro gepackt wird (Listing \ref{lst:makro}).

Siehe dazu auch \texttt{:h vim-arguments}.

Noch offen: Unterschiede vim, gvim? Wieso werden in MacVim die Menüs nicht angezeigt? Namenskonvention?

\section{Makros}

\subsection{Spezielles Mapping IMAP}
Die LS kommt mit einer speziellen Mapping-Technik, die sich vom normalen Mapping im Einfüge-Modus unterscheidet.
Grundsätzliches dazu findet sich mit \texttt{:h map-overviw} oder im File \texttt{\textasciitilde/.vim/plugin/imaps.vim}. 

\subsection{Platzhalter}

Die Makros der LS benutzen ein geniales Platzhalter-System. Die Platzhalter sind dort, wo bei einem eingefügten Makro das
nächste, relevante Text eingefügt werden muss. Der Platzhalter besteht aus \texttt{<+} als <<öffnende>> Klammer und
\texttt{+>} als <<schliessende>> Klammer. Dazwischen kann eine kurzer Memotext stehen, der dem Platzhalter noch einen
Sinn gibt (\texttt{<+Titel+>}). Mit der Tastenkombination \texttt{Ctrl-J} kann dann vorwärts von Platzhalter zu
Platzhalter gehüpft werden. Mit einem aktivierten Platzhalter ist man im Select-Modus (\texttt{:h select-mode}).

Im Select-Modus kann gleich des gewünschte eigetippt werden. Der Select-Modus
kommt einem als Mac- oder Windows-User am bekanntesten vor: anlog zur Auswahl, die mit dem nächsten Tastendruck
<<überschrieben>> wird.

Eine geniale Einrichtung, die einem unheimlich viel Bewegungsbefehle spart.

\subsection{Makros ein- und ausschalten}

Über die Variable \texttt{Imap\_FreezeImap} lässt sich die Makro-Funktion ein- und ausschalten.

\begin{itemize}
  \item \texttt{:let b:Imap\_FreezeImap=1} schaltet die Makros im aktuellen Buffer aus
  \item \texttt{:let b:Imap\_FreezeImap=0} schaltet die Makros im aktuellen Buffer ein
  \item \texttt{:let b:Imap\_FreezeImap} zeigt den aktuellen Status an. Allerdings erst, wenn die Variable
    erstmalig initialisiert (d.h. auf 0 oder 1 gesetzt) wurde. Ansonsten erfolgt eine Fehlermeldung
    (E121: Undefined variable: b:Imap\_FreezeImap).
\end{itemize}

Mit \texttt{g:Imap\_FreezeImap} können die Makros systemweit ein- und ausgeschaltet werden.


\subsection{Makros für LaTeX-Umgebung}

\subsubsection{Umgebung oder Paket einfügen mit F5}
\label{sec:methodpressingffife}

Die Taste \texttt{F5} -- gedrückt im Einfüge- oder Normal-Modus
-- hat abängig davon, ob sie auf einer leeren oder mit einem
Umgebungs-Schlüsselwort versehen Zeile gedrückt wird unterschiedliche
Auswirkung. Des weiteren spiel es eine Rolle ob man sich in der Preambel oder
im Dokumentteil des LaTeX-File befindet.

Mit dem Cursor auf einer \emph{leeren} Zeile wird mit \texttt{F5} eine Liste mit
Schlüsselwörter angezeigt. Diese stellen entweder Paketnamen (in der
Präambel) oder LaTeX-Umgebungen (im Dokumentteil) dar. Die entsprechende
Nummer wird gewählt und mit \texttt{<CR>} quittiert. Der entsprechende Code
wird eingefügt.

Enthält die Zeile bereits ein Schlüsselwort wird dieses als Paketname oder
als Umgebungsname verwendet und mit dem entsprechende LaTeX-Code ergänzt
(Tabelle \ref{tab:tasteffuenf}).

\begin{table}
  \centering
  \begin{tabular}{l|ll}
    \hline
    		& in Präambel & im Dokumentteil \\
	\hline
	leere Zeile & Auswahl Paket & Auswahl Umgebung\\
	Schlüsselwort & verwende \texttt{usepackage} & verwende \texttt{begin \dots end}\\
	Visual-Mode & nichts & schliesst ausgewählter Text in die Umgebung ein\\
	Line Visual-Mode & nichts & schliesst ausgewählten Text in die Umgebung ein (zeilenweise) \\
  \hline
      \end{tabular}
  \caption{Funktion der Taste F5, abhängig von der Position der Eingabe}
  \label{tab:tasteffuenf}
\end{table}

\subsubsection{Umgebung einfügen mit <<Hotkeys>> Shift-F1 bis Shift-F4}

Mit den Tastenkombinationen Shift-F1 bis Shift-F4 können die wichtigsten Umgebungen
abgespeichert werden.

Das geschieht in der Datei \texttt{tex.vim} mit der Variablen \texttt{g:Tex\_HotKeyMappings}.
Ein Beispiel findet sich im Listing \ref{lst:hotkey}.

\begin{lstlisting}[float,caption={Beispiel für Variable \texttt{g:Tex\_HotKeyMappings} in der Datei \texttt{tex.vim} },label=lst:hotkey]
" Nur die ersten 4 Umgebungen werden verwendet
" Shift-F1 bis Shift-F4
let g:Tex_HotKeyMappings = 'itemize,enumerate,frame,verbatim'
\end{lstlisting}

Im Gegensatz zum Einfügen einer Umgebung mit der F5-Taste funktionieren die Hotkeys nur im Einfüge-Modus.

\subsubsection{Drei-Buchstaben-Kürzel}

Umgebungen können im Einfüge-Modus mit einem Drei-Buchstaben-Kürzel eingesetzt werden.
Diese sind in Grossbuchstaben und beginnen mit \texttt{E}, gefolgt von zwei weiteren Grossbuchstaben
die 

\begin{itemize}
  \item bei Umgebungsbezeichnungen, die aus zwei Wörtern bestehen, jeweils die Anfangsbuchstaben darstellen
    (\texttt{\textbf{f}lush\textbf{l}eft} wird zu E\textbf{FL}).
  \item bei Umgebungsbezeichnungen, die aus einem Wort bestehen, den ersten und zweiten Buchstaben darstellen
    (\texttt{equation} wird zu EEQ).
  \item bei Konflikten wird nicht der zweite Buchstabe, sondern der \emph{letzte} Buchstabe genommen
    (\texttt{quote} wird zu EQE, \texttt{quotation} wird zu EQN).
\end{itemize}

\subsubsection{Umgebungen einsetzen im Visual-Modus: mit F5}

%siehe 3.1.2. Enclosing in Environments.
Die \LS bietet die Möglichkeit, ein im Visual-Modus ausgewählten Text automatisch in eine LaTeX-Umgebung zu fassen.
Die angezeigte Liste kann mit der Variable \texttt{g:Tex\_PromptedEnvironments} angepasst werden.

\subsubsection{Umgebung einsetzen im Visual-Modus mit Drei-Buchstaben-Kürzel}

Der Text kann auch ausgewählt werden und einem Drei-Buchstaben-Kürzel ausgewählt werden.
Die drei Buchstaben bestehen aus , (Komma) und den oben beschriebener Abkürzung, aber in \emph{Kleinbuchstaben}.

Aus \texttt{ETT} wird im Visual-Mode also \texttt{,tt}.
Danach ist man im Normal-Modus.

\subsubsection{Umgebung wechseln mit Shift-F5}

Wird im Normal-Modus die Shift-F5 gedrückt, kontrolliert \LS, in welcher Umgebung der Cursor aktuell ist.
Es kann so die Umgebung gewechselt werden.
Bei geschachtelten Umgebungen wird die innerste ersetzt.
Mit u kommt man wieder in die alte Umgebung.

Die mit Shift-F5 angezeigten, voreingestellten Umgebungen verändert werden
können wird im Abschnitt \ref{sec:methodpressingffife} dargestellt.

Bug? Bei mir wird bei Anwendung von Shift-F5 jeweils korrekterweise die innerste Umgebung ersetzt.
Allerdings werden die Backlash-Zeichen vor begin und end verdoppelt.

\subsubsection{Makros für LaTeX-Befehle anpassen}
\label{sec:commandmappings}

Die Variable \texttt{g:Tex\_PromptedEnvironments} bestimmt, welche Umgebungen mit \texttt{F5}
angezeigt wird.
Beispiel siehe Listing \ref{lst:environments}.

% "Blutte" Sonderzeichen in einer lstlisting-Umgebung gibt Fehlermeldung?
% können -> koennen im untenstehenden Beispiel.
\begin{lstlisting}[float,caption={Beispiel für Variable \texttt{g:Tex\_PromptedEnvironments} in der Datei \texttt{tex.vim} },label=lst:environments]
" Beliebig viele LaTeX-Umgebungen koennen angegeben werden.
let g:Tex_PromptedEnvironments = 'frame,center,lstlisting,itemize'
\end{lstlisting}

\section{Diverse Probleme}

\subsection{Folding ein- und ausschalten}

Siehe: http://stackoverflow.com/questions/3322453/how-can-i-disable-code-folding-in-vim-with-vim-latex

\subsection{Viewer in MacVim}

\subsubsection{Skim}

\subsubsection{Preview}

\subsubsection{PDF Reader (Acrobat}


http://www.iskysoft.de/edit-pdf/pdf-reader-for-mac.html


% http://stackoverflow.com/questions/12650528/viewing-pdfs-with-vim-latex-suite-start-preview-shell-returned-127
% https://groups.google.com/forum/#!topic/vim_mac/IO7oJCVxkQY
% http://reference-man.blogspot.co.uk/2011/09/fully-integrated-latex-in-macvim.html 
% \bibliographystyle{apacite}
% \bibliography{<++>}


\end{document}

